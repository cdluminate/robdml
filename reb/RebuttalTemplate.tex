\documentclass[10pt,twocolumn,letterpaper]{article}
\usepackage[rebuttal]{cvpr}

% Include other packages here, before hyperref.
\usepackage{graphicx}
\usepackage{amsmath}
\usepackage{amssymb}
\usepackage{booktabs}
\usepackage{microtype}
\usepackage{soul}
\usepackage{xpatch}

% If you comment hyperref and then uncomment it, you should delete
% egpaper.aux before re-running latex.  (Or just hit 'q' on the first latex
% run, let it finish, and you should be clear).
\usepackage[pagebackref,breaklinks,colorlinks,bookmarks=false]{hyperref}

%%START
%% https://tex.stackexchange.com/questions/378160/how-to-customize-bibtex-numbering-in-a-document
\makeatletter
\xpatchcmd{\@bibitem}
%  {\item}
  {\H@item}
  {\item[\@biblabel{\changekey{#1}}]}
  {}{}
\xpatchcmd{\@bibitem}
  {\the\value{\@listctr}}
  {\changekey{#1}}
  {}{}
\makeatother

\ExplSyntaxOn
\cs_new:Npn \changekey #1
 {
  \str_case:nVF {#1} \g_changekey_list_tl { ?? }
 }
\cs_new_protected:Npn \setchangekey #1 #2
 {
  \tl_gput_right:Nn \g_changekey_list_tl { {#1}{#2} }
 }
\tl_new:N \g_changekey_list_tl
\cs_generate_variant:Nn \str_case:nnF { nV }
\ExplSyntaxOff

\setchangekey{currat}{10}
\setchangekey{smoothat}{29}
%%END

% Support for easy cross-referencing
\usepackage[capitalize]{cleveref}
\crefname{section}{Sec.}{Secs.}
\Crefname{section}{Section}{Sections}
\Crefname{table}{Table}{Tables}
\crefname{table}{Tab.}{Tabs.}

% If you wish to avoid re-using figure, table, and equation numbers from
% the main paper, please uncomment the following and change the numbers
% appropriately.
%\setcounter{figure}{2}
%\setcounter{table}{1}
%\setcounter{equation}{2}

% If you wish to avoid re-using reference numbers from the main paper,
% please uncomment the following and change the counter for `enumiv' to
% the number of references you have in the main paper (here, 6).
%\let\oldthebibliography=\thebibliography
%\let\oldendthebibliography=\endthebibliography
%\renewenvironment{thebibliography}[1]{%
%     \oldthebibliography{#1}%
%     \setcounter{enumiv}{6}%
%}{\oldendthebibliography}


%%%%%%%%% PAPER ID  - PLEASE UPDATE
\def\cvprPaperID{201}% *** Enter the CVPR Paper ID here
\def\confName{CVPR}
\def\confYear{2022}

\begin{document}

%%%%%%%%% TITLE - PLEASE UPDATE
\title{Enhancing Adversarial Robustness for Deep Metric Learning -- Author Response}  % **** Enter the paper title here

\maketitle
\thispagestyle{empty}
\appendix

%%%%%%%%% BODY TEXT - ENTER YOUR RESPONSE BELOW
\section*{To Reviewer \#1}

\noindent\textbf{Q1.}
%
\ul{
The DML models are not special in terms of robustness against
adversarial attacks compared to other deep models.
}

\textbf{A:} 
%
We agree with the reviewer. Adversarial robustness at the visual representation
level remains to be an open but insightful problem. Since DML can be
interpreted as ``representation learning in metric space'', we believe the
defense for DML is one step closer to the representation-level robustness,
which is disentangled from specific vision tasks.

\noindent\textbf{Q2.}
%
\ul{
The relationship between the proposed method and a series of
adversarial training algorithms.
}

\cite{smoothat}

I’m hoping that the author could complete it in the related works section,
including reference 29. “Sat: Improving adversarial training via
curriculum-based loss smoothing” in the original manuscript and some other
studies, such as [1] and [2]. I think the idea of the proposed method bears a
certain similarity to these.

[1]. Cai, Qi-Zhi, et al. "Curriculum adversarial training." arXiv preprint arXiv:1805.04807 (2018).
\cite{currat}

[2]. Zhu, Xiaojin. "An optimal control view of adversarial machine learning." arXiv preprint arXiv:1811.04422 (2018).

\section*{To Reviewer \#2}

\noindent\textbf{Q1.}
%
\ul{
Sentences tend to be too long and complex; Method abbreviations are confusing
and hard to remember.
}

\noindent\textbf{Q2.}
%
\ul{
Tables are hard to grasp at a first glance, too many abbreviations and
notations. It would be useful to make captions self explanatory and refresh the
notation in them so that I dont have to search for them in different places of
the paper. Also, some color coding or marking the most important results would
be useful it is hard to ONE table with 20 rows and 20 columns, even harder with
5 such tables.
}

\noindent\textbf{Q3.}
%
\ul{
Paragraph on memory and time complexity between methods missing, are they all
equal?
}

\section*{To Reviewer \#3}

\noindent\textbf{Q1.}
%
\ul{
Robustness is important and many novel methods have been proposed, e.g., TRADES and AWP. I wonder whether these methods are not suitable for the metric learning task, and why the robustness of the metric learning task is unique.
}

\noindent\textbf{Q2.}
%
The main motivation of this work is that previous works suffer from excessively hard examples and this work aims to control the hardness. Does simple tricks like mixup work for adjusting the hardness.

\noindent\textbf{Q3.}
%
Tables (e.g., Tab.3) is hard to read. The authors may need to re-organize them and highlight somewhere.

\noindent\textbf{Q4.}
%
The performance on benign examples seems to drop a lot. Does it mean the HM is just another trade-off way?

%%%%%%%%% REFERENCES
{\small
\bibliographystyle{ieee_fullname}
\bibliography{egbib}
}

\end{document}
