% CVPR 2022 Paper Template
% based on the CVPR template provided by Ming-Ming Cheng (https://github.com/MCG-NKU/CVPR_Template)
% modified and extended by Stefan Roth (stefan.roth@NOSPAMtu-darmstadt.de)

\documentclass[10pt,twocolumn,letterpaper]{article}

%%%%%%%%% PAPER TYPE  - PLEASE UPDATE FOR FINAL VERSION
\usepackage[review]{cvpr}      % To produce the REVIEW version
%\usepackage{cvpr}              % To produce the CAMERA-READY version
%\usepackage[pagenumbers]{cvpr} % To force page numbers, e.g. for an arXiv version

% Include other packages here, before hyperref.
\usepackage{graphicx}
\usepackage{amsmath}
\usepackage{amssymb}
\usepackage{booktabs}

\usepackage{microtype}
\usepackage{xcolor}
\newcommand{\oo}[1]{\textcolor{orange}{#1}}

% It is strongly recommended to use hyperref, especially for the review version.
% hyperref with option pagebackref eases the reviewers' job.
% Please disable hyperref *only* if you encounter grave issues, e.g. with the
% file validation for the camera-ready version.
%
% If you comment hyperref and then uncomment it, you should delete
% ReviewTempalte.aux before re-running LaTeX.
% (Or just hit 'q' on the first LaTeX run, let it finish, and you
%  should be clear).
\usepackage[pagebackref,breaklinks,colorlinks]{hyperref}


% Support for easy cross-referencing
\usepackage[capitalize]{cleveref}
\crefname{section}{Sec.}{Secs.}
\Crefname{section}{Section}{Sections}
\Crefname{table}{Table}{Tables}
\crefname{table}{Tab.}{Tabs.}


%%%%%%%%% PAPER ID  - PLEASE UPDATE
\def\cvprPaperID{201} % *** Enter the CVPR Paper ID here
\def\confName{CVPR}
\def\confYear{2022}


\begin{document}

%%%%%%%%% TITLE - PLEASE UPDATE
\title{Enhancing Adversarial Robustness for Deep Metric Learning}

\author{Mo Zhou ~~~~~~~ Vishal M. Patel\\
Johns Hopkins University\\
{\tt\small mzhou32@jhu.edu ~~ vpatel36@jhu.edu}
% For a paper whose authors are all at the same institution,
% omit the following lines up until the closing ``}''.
% Additional authors and addresses can be added with ``\and'',
% just like the second author.
% To save space, use either the email address or home page, not both
}
\maketitle

%%%%%%%%% ABSTRACT
\begin{abstract}
	Deep metric learning models, being vulnerable to adversarial
	attacks, have safety and security implications in applications.
	%
	Improving the adversarial robustness of a model against the attacks becomes
	an important problem.
	%
	The existing defense methods are based on adversarial training.
	%
	However, they \oo{(1) robustness (2) speed/cost. (3) sampling matters.}
	%
	In this paper, we \oo{TBD}.
	%
	The proposed method is validated on three commonly
	used deep metric learning datasets, namely CUB-200-2011, Cars-196,
	and Stanford Online Product.
	%
	Comprehensive experimental results show our method outperforms
	the existing defense methods.
\end{abstract}

%%%%%%%%% BODY TEXT
\section{Introduction}
\label{sec:1}

\section{Related Works}
\label{sec:2}

\section{Our Approach}
\label{sec:3}

\section{Experiments}
\label{sec:4}

\section{Discussions}
\label{sec:5}

\section{Conclusion}
\label{sec:6}

\cite{Authors14}

%%%%%%%%% REFERENCES
{\small
\bibliographystyle{ieee_fullname}
\bibliography{egbib}
}

\end{document}
