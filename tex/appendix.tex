% Copyright (C) 2021-2022 Mo Zhou <cdluminate@gmail.com>
% License: CC-BY-NC-SA-4.0
% Preprint: https://arxiv.org/abs/2203.01439
% Preprint License: CC-BY-NC-SA-4.0

\clearpage
\appendix

\setcounter{figure}{8}
\setcounter{table}{6}

\section*{Supplementary Material}

In this supplementary material, we provide more technical details and
additional discussions that are excluded from the manuscript due to space
limit.

\tableofcontents

\section{Additional Information}
\label{sec:a}

\subsection{Potential Societal Impact}
\label{sec:a1}

\noindent\textbf{I. Security.}

Adversarial defenses alleviate the negative societal impact of adversarial
attacks, and hence have positive societal impact.

\subsection{Limitations of Our Method}
\label{sec:a2}

\noindent\textbf{I. Assumptions.}

~\newline
\noindent \ul{(1) Triplet Training Assumption.}

Our method assumes sample triplets are used for training.
%
Our method may not be compatible to other non-triplet DML loss functions.
%
Adversarial training with other DML loss functions is left for future study.

~\newline
\noindent \ul{(2) Embedding Space Assumption.}

We follow the common setups~\cite{revisiting,robrank} on the embedding space.
%
Namely, (1) the embedding vectors are normalized onto the real unit
hypersphere;
%
(2) the distance function $d(\cdot,\cdot)$ is Euclidean distance.
%
Our formulations are developed upon the two assumptions.
%
It is unknown whether our method method will be effective when embedding
vectors are \emph{not} normalized.
%
And it is unknown whether our method will be effective when $d(\cdot,\cdot)$ is
replaced as other distance metrics, \eg, cosine distance.

~\newline
\noindent \ul{(3) Optimizer Assumption.}

Our method assumes PGD~\cite{madry} is used for optimizing the HM objective
to create adversarial examples.
%
The Eq.~(4)-(5) may not necessarily hold with other possible optimizers.

~\newline
\noindent\textbf{II. Performance-Sensitive Factors.}

~\newline
\noindent \ul{(1) Maximum number of PGD iterations $\eta$.}

Our method's sensitivity to $\eta$ has been demonstrated by Tab.~3-6 and
Fig.~6-8.
%
A larger $\eta$ indicates higher training cost, and stronger adversarial
examples are created for adversarial training.
%
As a result, a larger $\eta$ leads to a higher robustness (ERS) and a lower
R@1 performance.
%
Our method consistently achieves a higher ERS under different $\eta$ settings
compared to previous methods, and hence are the most efficient defense method.
%
Experiments with $\eta$ larger than $32$ are not necessary because ERS plateaus
according to Fig.~6-8.

~\newline
\noindent \ul{(2) Source hardness $H_\mathsf{S}$ and destination hardness
$H_\mathsf{D}$.}

The $H_\mathsf{S}$ and $H_\mathsf{D}$ are the only two adjustable items in HM.

The source hardness $H_\mathsf{S}$ depends on triplet sampling strategy.
%
We conduct experiments with existing triplet sampling strategies in order to
focus on defense.

The choices for $H_\mathsf{D}$ are more flexible than those of $H_\mathsf{S}$,
as discussed in Sec.~3.1.
%
In the experiments, we study some possible choices following the discussion and
design LGA based on the empirical observations.

~\newline
\noindent\ul{(3) Parameters involved in $g_\mathsf{LGA}$.}

\begin{table*}
\resizebox{\linewidth}{!}{
\setlength{\tabcolsep}{0.47em}
\renewcommand{\arraystretch}{1.1}%
\begin{tabular}{c|cc|cccc|ccccc|ccccc|c}
	\toprule
\multirow{2}{*}{\textbf{Dataset}} & \multirow{2}{*}{\textbf{Defense}} & \multirow{2}{*}{$\eta$} & \multicolumn{4}{c|}{\textbf{Benign Example}} & \multicolumn{10}{c|}{\textbf{White-Box Attacks for Robustness Evaluation}} & \multirow{2}{*}{\textbf{ERS$\uparrow$}}\tabularnewline
\cline{4-17} \cline{5-17} \cline{6-17} \cline{7-17} \cline{8-17} \cline{9-17} \cline{10-17} \cline{11-17} \cline{12-17} \cline{13-17} \cline{14-17} \cline{15-17} \cline{16-17} \cline{17-17}
 &  &  & R@1$\uparrow$ & R@2$\uparrow$ & mAP$\uparrow$ & NMI$\uparrow$ & CA+$\uparrow$ & CA-$\downarrow$ & QA+$\uparrow$ & QA-$\downarrow$ & TMA$\downarrow$ & ES:D$\downarrow$ & ES:R$\uparrow$ & LTM$\uparrow$ & GTM$\uparrow$ & GTT$\uparrow$ & \tabularnewline
 \midrule
\multirow{2}{*}{CUB} & HM{[}$\mathcal{S},g_{\mathsf{LGA}}${]} & 8 & 38.0 & 48.3 & 21.8 & 49.3 & 12.7 & 46.4 & 11.6 & 39.9 & 0.567 & 0.783 & 16.8 & 11.9 & 27.9 & 1.4 & 32.4\tabularnewline
 & HM{[}$\mathcal{S},g_{\mathsf{LGA}}${]} ($u=2.2$) & 8 & 34.8 & 45.5 & 15.2 & 47.1 & 13.4 & 36.0 & 17.2 & 26.1 & 0.934 & 0.244 & 20.1 & 15.9 & 27.3 & 3.8 & 36.1\tabularnewline
 \bottomrule
\end{tabular}}
	\caption{The efficacy of parameter $u$ for clipping loss value $\ell_{t-1}$.
	%
	Stronger adversarial examples will be created for training if the loss value is not
	clipped (equivalent to setting $u$ to the theoretical upper bound of loss,
	\ie, 2.2).
	}
	\label{tab:u}
\end{table*}


A constant $u$ is used to normalize the loss value of the previous training
iteration $\ell_{t-1}$ into $\bar{\ell}_{t-1}\in[0,1]$.
%
The constant is empirically selected as $u=\gamma$ in our experiment.
%
According to our observation, the loss value will quickly decrease below
$\gamma$, and will remain in the $[0,\gamma]$ range for the whole training
process.
%
If we set $u$ to a larger constant than $\gamma$, the normalized loss
$\bar{\ell}_{t-1}$ will be smaller, and results in stronger adversarial
examples through HM[$\mathsf{S},g_\mathsf{LGA}$] and harms the model performance on benign examples,
as shown in \cref{tab:u}.

Another parameter in $g_\mathsf{LGA}$ is the triplet margin parameter $\gamma$
in order to align to the hardness range of Semihard triplets, \ie, $-\gamma<g_\mathsf{LGA}<0$.
%
We follow the common setup~\cite{revisiting} for this parameter.

~\newline
\noindent\ul{(4) Constant parameter $\lambda$ for ICS loss term $L_\text{ICS}$.}

The weight constant $\lambda$ is set as $0.5$ by default, and $0.05$ on the SOP dataset.
%
As demonstrated in Tab.~5, there is a trade-off between robustness and
performance on benign examples when tuning the $\lambda$ parameter.

Additional experiments with $\lambda=0.5$ on the SOP dataset can be found in \cref{tab:soplambda}.
%
Our method is sensitive to this parameter.
An excessively large $\lambda$ on the SOP datasets leads to worse performance on benign examples
and worse robustness.

\begin{table*}
\resizebox{\linewidth}{!}{
\setlength{\tabcolsep}{0.42em}
\renewcommand{\arraystretch}{1.1}%
\begin{tabular}{c|cc|cccc|ccccc|ccccc|c}
	\toprule
\multirow{2}{*}{\textbf{Dataset}} & \multirow{2}{*}{\textbf{Defense}} & \multirow{2}{*}{$\eta$} & \multicolumn{4}{c|}{\textbf{Benign Example}} & \multicolumn{10}{c|}{\textbf{White-Box Attacks for Robustness Evaluation}} & \multirow{2}{*}{\textbf{ERS$\uparrow$}}\tabularnewline
\cline{4-17} \cline{5-17} \cline{6-17} \cline{7-17} \cline{8-17} \cline{9-17} \cline{10-17} \cline{11-17} \cline{12-17} \cline{13-17} \cline{14-17} \cline{15-17} \cline{16-17} \cline{17-17}
 &  &  & R@1$\uparrow$ & R@2$\uparrow$ & mAP$\uparrow$ & NMI$\uparrow$ & CA+$\uparrow$ & CA-$\downarrow$ & QA+$\uparrow$ & QA-$\downarrow$ & TMA$\downarrow$ & ES:D$\downarrow$ & ES:R$\uparrow$ & LTM$\uparrow$ & GTM$\uparrow$ & GTT$\uparrow$ & \tabularnewline
 \midrule
\multirow{2}{*}{SOP} & HM{[}$\mathcal{S},g_{\mathsf{LGA}}${]}\&ICS ($\lambda=0.5$) & 8 & 42.4 & 47.1 & 10.2 & 84.2 & 34.9 & 3.8 & 36.7 & 2.3 & 0.879 & 0.093 & 35.3 & 36.5 & 35.2 & 49.1 & 60.1\tabularnewline
 & HM{[}$\mathcal{S},g_{\mathsf{LGA}}${]}\&ICS ($\lambda=0.5$) & 32 & 41.5 & 46.1 & 9.9 & 84.1 & 36.1 & 3.1 & 37.6 & 2.1 & 0.873 & 0.086 & 35.7 & 36.8 & 34.7 & 50.2 & 60.8\tabularnewline
 \bottomrule
\end{tabular}}
	\caption{An excessively large $\lambda$ may lead to worse performance and robustness.}
	\label{tab:soplambda}
\end{table*}

~\newline
\noindent\ul{(5) Backbone deep neural network.}

We adopt ResNet-18 following the state-of-the-art defense~\cite{robrank} for
fair comparison.
%
Since our proposed method is independent to the backbone choice, and hence
is expected to be effective with different backbone models.

\subsection{Use of Existing Assets}

\noindent\ul{(1) Datasets.}

All the datasets used in our paper are public datasets,
and their corresponding papers are cited.
%
The CUB~\cite{cub200} dataset includes images of birds.
%
The CARS~\cite{cars196} dataset includes images of cars.
%
The SOP~\cite{sop} dataset includes images of online products.

~\newline
\noindent\ul{(2) Code and Implementation.}

Our implementation is built upon PyTorch
and the public code of the state-of-the-art defense method ACT~\cite{robrank}
(License: Apache-2.0).

\section{Technical Details \& Minor Discussions}
\label{sec:b}

\subsection{Difference between Existing Defenses \& HM}
\label{sec:b1}

\noindent\textbf{I. Embedding-Shifted Triplet (EST).}~\cite{advrank}

Embedding-Shifted Triplet (EST)~\cite{advrank} adopts adversarial counterparts
of $\va,\vp,\vn$ with maximum embedding move distance off their original
locations, \ie,
%
\begin{equation}
%
L_\text{EST}=L_\text{T}(\tilde{\va},\tilde{\vp},\tilde{\vn};\gamma)
%
\end{equation}
%
where
$\tilde{\va}=\phi(\mA+\vr^*)$, and $\vr^*=\arg\max_{\vr}d_\phi(\mA+\vr, \mA)$.
%
The $\tilde{\vp}$ and $\tilde{\vn}$ are obtained similarly.

\ul{(1) Relationship with HM:}

Since EST only aims to maximize the embedding move distance off its original
location without specifying any direction, it leads to a random hardness value.
%
The expectation $E[H(\cdot)]$ of its resulting adversarial triplet is expected
to be close to $E[H_\mathsf{S}]$.
%
Because the perturbed triplet can be either harder or easier than the benign
triplet.
%
Namely, EST merely indirectly increase the hardness of the training triplet,
and may even decrease its hardness.
%
Thus, EST suffers from inefficiency in adversarial training compared to HM.

~\newline
\noindent\textbf{II. Anti-Collapse Triplet (ACT).}~\cite{robrank}

Anti-Collapse Triplet (ACT)~\cite{robrank} ``collapses'' the embedding vectors
of positive and negative sample, and enforces the model to separate them apart,
\ie,
%
\begin{align}
%
	L_\text{ACT} &=L_\text{T}(\va, \overrightarrow{\vp},
\overleftarrow{\vn};\gamma),\\
%
	[\overrightarrow{\vp},\overleftarrow{\vn}] & =[\phi(\mP+\vr_p^*), \phi(\mN+\vr_n^*)]\\
%
	[\vr_p^*,\vr_n^*] &= \argmin_{\vr_p,\vr_n} d_\phi(\mP+\vr_p, \mN+\vr_n).
%
\end{align}

\ul{(1) Relationship with HM:}

When ACT successively ``collapses'' the positive and negative embedding
vectors together, the hardness will be zero, \ie, $E[H(\cdot)]=0$.
%
But ACT is not equivalent to $HM[\cdot,0]$ because the two methods have
different objectives and use different gradients.
%
Besides, in order to avoid the ``misleading gradients''~\cite{robrank}, ACT
fixes the anchor and only perturb the positive and negative samples, which
makes the objective for creating adversarial examples more difficult to
optimize in practice.
%
In brief, ACT is also indirectly increasing the loss value, suffering from
inefficient adversarial learning.

\begin{table*}[t]
\resizebox{\linewidth}{!}{
\setlength{\tabcolsep}{0.47em}
\renewcommand{\arraystretch}{1.1}%
\begin{tabular}{c|cc|cccc|ccccc|ccccc|c}
	\toprule
\multirow{2}{*}{\textbf{Dataset}} & \multirow{2}{*}{\textbf{Defense}} & \multirow{2}{*}{$\eta$} & \multicolumn{4}{c|}{\textbf{Benign Example}} & \multicolumn{10}{c|}{\textbf{White-Box Attacks for Robustness Evaluation}} & \multirow{2}{*}{\textbf{ERS$\uparrow$}}\tabularnewline
\cline{4-17} \cline{5-17} \cline{6-17} \cline{7-17} \cline{8-17} \cline{9-17} \cline{10-17} \cline{11-17} \cline{12-17} \cline{13-17} \cline{14-17} \cline{15-17} \cline{16-17} \cline{17-17}
 &  &  & R@1$\uparrow$ & R@2$\uparrow$ & mAP$\uparrow$ & NMI$\uparrow$ & CA+$\uparrow$ & CA-$\downarrow$ & QA+$\uparrow$ & QA-$\downarrow$ & TMA$\downarrow$ & ES:D$\downarrow$ & ES:R$\uparrow$ & LTM$\uparrow$ & GTM$\uparrow$ & GTT$\uparrow$ & \tabularnewline
 \midrule
\multirow{3}{*}{CUB} & HM{[}$\mathcal{S},g_{\mathsf{1/2}}${]} & 8 & 38.7 & 48.5 & 22.0 & 49.2 & 12.6 & 48.6 & 12.3 & 41.3 & 0.562 & 0.825 & 13.5 & 12.9 & 26.9 & 1.8 & 31.7\tabularnewline
& HM{[}$\mathcal{S},g_{\mathsf{LGA}}${]} & 8 & 38.0 & 48.3 & 21.8 & 49.3 & 12.7 & 46.4 & 11.6 & 39.9 & 0.567 & 0.783 & 16.8 & 11.9 & 27.9 & 1.4 & 32.4\tabularnewline
& HM{[}$\mathcal{S},g_{\mathsf{2}}${]} & 8 & 37.4 & 48.2 & 17.1 & 49.2 & 12.9 & 45.1 & 13.2 & 39.1 & 0.599 & 0.738 & 17.7 & 12.8 & 27.5 & 1.9 & 33.1\tabularnewline
\bottomrule
\end{tabular}}
	\caption{Non-linear Gradual Adversary Examples.}
	\label{tab:nonlinga}
\end{table*}


~\newline\textbf{III. Min-max Adversarial Training with Triplet Loss.}

The direct formulation of min-max adversarial training~\cite{madry} for
triplet loss-based DML is:
%
\begin{equation}
%
	\theta^* = \argmin_\theta [ \argmax_{\vr_a, \vr_p, \vr_n} L_\text{T}(
	\mA + \vr_a, \mP + \vr_p, \mN + \vr_n)]
%
\end{equation}
%
Previous works~\cite{advrank,robrank} point out this method will easily lead
to model collapse.
%
Our observation suggests the same.

\ul{(1) Relationship with HM:}

Maximizing $L_\text{T}$ is equivalent to maximizing $\tilde{H}_\mathsf{S}$.
%
This can be expressed as HM[$H_\mathsf{S},2$] as discussed in Sec.~3.1.

\subsection{Hardness Manipulation (HM)}
\label{sec:b2}

\noindent\textbf{I. Adjustable Items}

The only two adjustable items in HM are $H_\mathsf{S}$ and $H_\mathsf{D}$.
%
They are discussed in the ``Performance-Sensitive Factors'' part of the
previous section.


~\newline
\noindent\textbf{II. Extreme Values of Hardness.}

As reflected in Sec.~3.1 and Fig.~2, the range of $H(\cdot)$ is $[-2, 2]$.
%
The embedding vectors have been normalized to the real unit hypersphere
as pointed out in the manuscript.
%
And the range of distance between any two points on the hypersphere is
$[0,2]$.
%
Hence the extreme values are:
%
\begin{align}
	& \max H(\mA, \mP, \mN)\\
	=& \max [d_\phi(\mA, \mP) - d_\phi(\mA, \mN)]\\
	=& \max [d_\phi(\mA, \mP)] - \min [d_\phi(\mA, \mN)]\\
	=& 2 - 0,
\end{align}
%
\begin{align}
	& \min H(\mA, \mP, \mN)\\
	=& \min [d_\phi(\mA, \mP) - d_\phi(\mA, \mN)]\\
	=& \min [d_\phi(\mA, \mP)] - \max [d_\phi(\mA, \mN)]\\
	=& 0 - 2.
\end{align}
%
Namely $H(\cdot)\in[-2,2]$. Meanwhile, since
%
\begin{equation}
%
L_\text{T}(\mA, \mP, \mN; \gamma) = \max(0, H(\mA, \mP, \mN) + \gamma),
%
\end{equation}
%
we have $L_\text{T}\in[0, 2+\gamma]$.

\subsection{Gradual Adversary}
\label{sec:b3}

\noindent\textbf{I. Parameters}

The parameters, namely $u$ and $\gamma$ are discussed in the previous section
of this supplementary material, see ``Performance-Sensitive Factors''.

The parameter $\xi$ in $g_\mathsf{B}$ is set as $0.1$, but it is not an important parameter.
%
The function $g_\mathsf{B}$ is only used for demonstrating that ``slightly
boosting the destination hardness can further increase ERS'' as discussed
in Sec.~3.1.


~\newline
\noindent\textbf{II. Non-linear Gradual Adversary}

In Sec.~3.2, more complicated designs are left for future work.
%
We provide two Non-linear Gradual Adversary examples, namely
$g_\mathsf{2}$ and $g_\mathsf{1/2}$, as follows:
%
\begin{align}
	g_\mathsf{2}(\cdot)   &= -\gamma \cdot (\bar{\ell}_{t-1})^{2}  &~ \in [-\gamma,0]\\
	g_\mathsf{1/2}(\cdot) &= -\gamma \cdot (\bar{\ell}_{t-1})^{1/2} &~ \in [-\gamma, 0]
\end{align}
%
Compared to LGA, $g_\mathsf{2}$ is more ``eager'' to result in strong adversarial
examples in the early phase of training, while $g_\mathsf{1/2}$ is more ``conservative''
in creating strong adversarial examples in the early phase of training
(adversarial examples from HM are stronger if the function value is closer to $0$).
%
The corresponding experiments can be found in \cref{tab:nonlinga}.

\begin{table*}
\resizebox{\linewidth}{!}{
\setlength{\tabcolsep}{0.47em}
\renewcommand{\arraystretch}{1.1}%
\begin{tabular}{c|cc|cccc|ccccc|ccccc|c}
	\toprule
\multicolumn{1}{c|}{\multirow{2}{*}{\textbf{Dataset}}} & \multirow{2}{*}{\textbf{Defense}} & \multirow{2}{*}{$\eta$} & \multicolumn{4}{c|}{\textbf{Benign Example}} & \multicolumn{10}{c|}{\textbf{White-Box Attacks for Robustness Evaluation}} & \multirow{2}{*}{\textbf{ERS$\uparrow$}}\tabularnewline
\cline{4-17} \cline{5-17} \cline{6-17} \cline{7-17} \cline{8-17} \cline{9-17} \cline{10-17} \cline{11-17} \cline{12-17} \cline{13-17} \cline{14-17} \cline{15-17} \cline{16-17} \cline{17-17}
 &  &  & R@1$\uparrow$ & R@2$\uparrow$ & mAP$\uparrow$ & NMI$\uparrow$ & CA+$\uparrow$ & CA-$\downarrow$ & QA+$\uparrow$ & QA-$\downarrow$ & TMA$\downarrow$ & ES:D$\downarrow$ & ES:R$\uparrow$ & LTM$\uparrow$ & GTM$\uparrow$ & GTT$\uparrow$ & \tabularnewline
 \midrule
\multirow{2}{*}{CUB} & HM{[}$\mathcal{S},g_{\mathsf{LGA}}${]}\&ICS & 8 & 37.2 & 47.8 & 21.4 & 48.4 & 12.9 & 40.9 & 14.7 & 33.7 & 0.806 & 0.487 & 17.1 & 13.2 & 26.3 & 2.3 & 33.5\tabularnewline
& HM[$\mathcal{S},g_\mathsf{LGA}$]\&$L_\text{ICS}(\cdots;\gamma=0.2)$ & 8 & 35.8 & 46.6 & 16.4 & 48.3 & 13.2 & 39.9 & 12.7 & 33.3 & 0.775 & 0.507 & 15.9 & 14.9 & 27.2 & 2.7 & 33.6\tabularnewline
 \bottomrule
\end{tabular}}
	\caption{The margin parameter is set to $0$ in order to avoid negative effect.
	%
	R@1 performance drops with marginal robustness gain.}
	\label{tab:0margin}
\end{table*}

\subsection{Intra-Class Structure (ICS)}

\noindent\textbf{I. Parameters}

The ICS loss term can be appended to the loss for adversarial training.
%
The only parameter for ICS loss term is the weight constant $\lambda$,
and has been discussed in the ``Performance-Sensitive Factors'' part
of the previous section.

The margin parameter is set to $0$ in order to avoid negative effect in $L_\text{ICS}$,
as shown in \cref{tab:0margin}.

~\newline
\noindent\textbf{II. Gradients in Fig.~5 (a) in Manuscript}

According to \cite{robrank}, when the embedding vectors are normalized onto the
real unit hypersphere and Euclidean distance is used, the gradients of the
triplet loss with respect to the anchor, positive, and negative embedding
vectors are respectively:
%
\begin{align}
%
	\frac{\partial L_\text{T}}{\partial \va} &= \frac{\va - \vp}{\|\va - \vp\|}
	- \frac{\va - \vn}{\| \va - \vn \|} \\
	\frac{\partial L_\text{T}}{\partial \vp} &= \frac{\vp - \va}{\|\va - \vp\|} \\
	\frac{\partial L_\text{T}}{\partial \vn} &= \frac{\va - \vn}{\|\va - \vn\|},
%
\end{align}
%
when $L_\text{T}>0$.
And the above equations have been reflected in Fig.~5 (a) in terms of vector direction.

~\newline
\noindent\textbf{III. Alternative Design for Exploiting Sextuplet}

Let $\tilde{\va}$, $\tilde{\vp}$, and $\tilde{\vn}$ be $\phi(\mA+\hat{\vr}_a)$,
$\phi(\mP+\hat{\vr}_p)$ and $\phi(\mN+\hat{\vr}_n)$ respectively.
%
In our proposed ICS loss term, only $(\va, \tilde{\va}, \vp)$ are involved.
%
Other alternative selections of triplets from the sextuplet are possible,
but are not as effective as $L_\text{ICS}$ for improving adversarial robustness.

\ul{(1) $L_\text{T}(\va, \tilde{\va}, \tilde{\vp})$}

As shown in Fig.~5 (c) in the manuscript, the position of $\tilde{\vp}$ is
always further away from both $\va$ and $\tilde{\va}$ than $\vp$ due to
the gradient direction.
%
Thus, the loss value of $L_\text{T}(\va, \tilde{\va}, \tilde{\vp})$ will always
be smaller than $L_\text{ICS}$, and hence is less effective than $L_\text{ICS}$.

\ul{(2) $L_\text{T}(\va, \vp, \vn)$: Mixing regular training
and adversarial training.}

According to our observation, mixing regular training and adversarial training
leads to better R@1 with drastic robustness degradation for both ACT and
our defense.

\ul{(3) $L_\text{T}(\vp, \tilde{\vp}, \va)$: Symmetric counterpart of $L_\text{ICS}$.}

Every sample in the training dataset will be used as anchor for once per
epoch.
%
Such symmetric loss term is duplicated to $L_\text{ICS}$ and is
not necessary.
%
Experimental results suggest negligible difference compared to $L_\text{ICS}$.

\ul{(4) $L_\text{T}(\va, \tilde{\va}, \tilde{\vn})$: $\tilde{\vn}$ is very close to $\va$ in Fig.~5.}

It enforces \emph{inter}-class structure instead of intra-class structure.
%
Besides, experimental results suggest negligible difference.
%
We speculate this loss term is duplicated to the adversarial training loss term,
\ie, $L_\text{T}(\tilde{\va}, \tilde{\vp}, \tilde{\vn})$, which enforces
inter-class structure as well in a stronger manner.

\ul{(5) $L_\text{T}(\va, \vp, \tilde{\vn})$}

According to our observation, it leads to better R@1 performance on benign
examples, but drastically reduce the robustness.

\subsection{Experiments and Evaluation}

\noindent\textbf{I Hardware and Software Configuration}

We conduct the experiments with two Nvidia Titan Xp GPUs (12GB of memory each)
under the Ubuntu 16.04 operating system with PyTorch 1.8.2 in distributed data
parallel.

~\newline
\noindent\textbf{II. Training Cost}

In the manuscript, the training cost of adversarial training is caluclated as
$\eta+1$, which is the number of forward-backward propagation involved in each
iteration of the training process, in order to reflect the training efficiency
(gain as high robustness as possible given a fixed number of forward-backward
calculation for adversarial training) of different defense methods.
%
Specifically, PGD creates adversarial examples from the $\eta$ times of
forward-backward computation.
%
With the resulting adversarial examples, the network requires once
more forward-backward computation to update the model parameters.

~\newline
\noindent\textbf{III. Complete Results for Tab.~2 in Manuscript}

\begin{table*}
\resizebox{\linewidth}{!}{
\setlength{\tabcolsep}{0.3em}%
\begin{tabular}{c|cc|cccc|ccccc|ccccc|c}
\hline 
\multirow{2}{*}{\textbf{Dataset}} & \multirow{2}{*}{\textbf{Defense}} & \multirow{2}{*}{$\eta$} & \multicolumn{4}{c|}{\textbf{Benign Example}} & \multicolumn{10}{c|}{\textbf{White-Box Attacks for Robustness Evaluation}} & \multirow{2}{*}{\textbf{ERS}}\tabularnewline
\cline{4-17} \cline{5-17} \cline{6-17} \cline{7-17} \cline{8-17} \cline{9-17} \cline{10-17} \cline{11-17} \cline{12-17} \cline{13-17} \cline{14-17} \cline{15-17} \cline{16-17} \cline{17-17} 
 &  &  & R@1$\uparrow$ & R@2$\uparrow$ & mAP$\uparrow$ & NMI$\uparrow$ & CA+$\uparrow$ & CA-$\downarrow$ & QA+$\uparrow$ & QA-$\downarrow$ & TMA$\downarrow$ & ES:D$\downarrow$ & ES:R$\uparrow$ & LTM$\uparrow$ & GTM$\uparrow$ & GTT$\uparrow$ & \tabularnewline
\hline 
\hline 
\multirow{5}{*}{CUB} & HM{[}$\mathcal{R},\mathcal{R}${]} & N/A & 53.9 & 66.4 & 26.1 & 59.5 & 0.0 & 100.0 & 0.0 & 99.9 & 0.883 & 1.762 & 0.0 & 0.0 & 14.1 & 0.0 & 3.8\tabularnewline
 & HM{[}$\mathcal{R},\mathcal{M}${]} & 8 & 27.0 & 36.0 & 13.2 & 42.5 & 19.4 & 48.0 & 22.2 & 32.0 & 0.535 & 0.867 & 11.6 & 10.4 & 19.3 & 2.9 & 35.1\tabularnewline
 & HM{[}$\mathcal{R},\mathcal{S}${]} & 8 & C & o & l & l & a & p & s & e &  &  &  &  &  &  & \tabularnewline
 & HM{[}$\mathcal{R},\mathcal{D}${]} & 8 & C & o & l & l & a & p & s & e &  &  &  &  &  &  & \tabularnewline
 & HM{[}$\mathcal{R},\mathcal{H}${]} & 8 & C & o & l & l & a & p & s & e &  &  &  &  &  &  & \tabularnewline
\hline 
\multirow{5}{*}{CUB} & HM{[}$\mathcal{M},\mathcal{R}${]} & 8 & 43.9 & 55.9 & 27.1 & 54.2 & 0.2 & 100.0 & 0.4 & 99.6 & 0.849 & 1.640 & 0.2 & 0.1 & 19.1 & 0.0 & 5.4\tabularnewline
 & HM{[}$\mathcal{M},\mathcal{M}${]} & N/A & 44.0 & 56.1 & 27.2 & 54.6 & 0.1 & 100.0 & 0.4 & 99.8 & 0.843 & 1.717 & 0.3 & 0.3 & 18.4 & 0.0 & 5.0\tabularnewline
 & HM{[}$\mathcal{M},\mathcal{S}${]} & 8 & C & o & l & l & a & p & s & e &  &  &  &  &  &  & \tabularnewline
 & HM{[}$\mathcal{M},\mathcal{D}${]} & 8 & C & o & l & l & a & p & s & e &  &  &  &  &  &  & \tabularnewline
 & HM{[}$\mathcal{M},\mathcal{H}${]} & 8 & C & o & l & l & a & p & s & e &  &  &  &  &  &  & \tabularnewline
\hline 
\multirow{5}{*}{CUB} & HM{[}$\mathcal{S},\mathcal{R}${]} & 8 & 48.3 & 60.1 & 30.3 & 56.3 & 2.0 & 84.4 & 1.4 & 85.4 & 0.835 & 0.895 & 6.1 & 1.6 & 21.0 & 0.0 & 13.7\tabularnewline
 & HM{[}$\mathcal{S},\mathcal{M}${]} & 8 & 38.4 & 49.7 & 22.9 & 50.3 & 10.9 & 50.5 & 10.8 & 44.6 & 0.680 & 0.722 & 13.3 & 11.2 & 25.8 & 1.2 & 29.6\tabularnewline
 & HM{[}$\mathcal{S},\mathcal{S}${]} & N/A & 55.7 & 68.2 & 35.8 & 61.0 & 0.0 & 100.0 & 0.0 & 99.6 & 0.942 & 1.272 & 0.0 & 0.0 & 19.2 & 0.0 & 6.2\tabularnewline
 & HM{[}$\mathcal{S},\mathcal{D}${]} & 8 & C & o & l & l & a & p & s & e &  &  &  &  &  &  & \tabularnewline
 & HM{[}$\mathcal{S},\mathcal{H}${]} & 8 & C & o & l & l & a & p & s & e &  &  &  &  &  &  & \tabularnewline
\hline 
\multirow{5}{*}{CUB} & HM{[}$\mathcal{D},\mathcal{R}${]} & 8 & 52.7 & 64.0 & 33.7 & 58.9 & 0.0 & 100.0 & 0.2 & 100.0 & 0.841 & 1.749 & 0.1 & 0.0 & 19.0 & 0.0 & 4.8\tabularnewline
 & HM{[}$\mathcal{D},\mathcal{M}${]} & 8 & 50.7 & 63.8 & 33.1 & 58.7 & 0.0 & 100.0 & 0.3 & 99.9 & 0.841 & 1.744 & 0.2 & 0.0 & 18.5 & 0.0 & 4.8\tabularnewline
 & HM{[}$\mathcal{D},\mathcal{S}${]} & 8 & C & o & l & l & a & p & s & e &  &  &  &  &  &  & \tabularnewline
 & HM{[}$\mathcal{D},\mathcal{D}${]} & N/A & 51.4 & 64.2 & 33.4 & 59.3 & 0.0 & 100.0 & 0.4 & 99.9 & 0.842 & 1.753 & 0.0 & 0.0 & 19.7 & 0.0 & 4.9\tabularnewline
 & HM{[}$\mathcal{D},\mathcal{H}${]} & 8 & 54.7 & 66.9 & 36.2 & 60.0 & 0.0 & 100.0 & 0.2 & 99.8 & 0.874 & 1.569 & 0.4 & 0.0 & 18.6 & 0.0 & 5.4\tabularnewline
\hline 
\multirow{5}{*}{CUB} & HM{[}$\mathcal{H},\mathcal{R}${]} & 8 & 51.0 & 63.5 & 32.9 & 59.2 & 0.0 & 100.0 & 0.2 & 99.9 & 0.837 & 1.750 & 0.0 & 0.0 & 17.7 & 0.0 & 4.7\tabularnewline
 & HM{[}$\mathcal{H},\mathcal{M}${]} & 8 & 52.2 & 64.0 & 33.1 & 58.4 & 0.0 & 100.0 & 0.2 & 100.0 & 0.841 & 1.750 & 0.0 & 0.0 & 18.9 & 0.0 & 4.8\tabularnewline
 & HM{[}$\mathcal{H},\mathcal{S}${]} & 8 & C & o & l & l & a & p & s & e &  &  &  &  &  &  & \tabularnewline
 & HM{[}$\mathcal{H},\mathcal{D}${]} & 8 & 52.6 & 64.9 & 34.4 & 60.0 & 0.0 & 100.0 & 0.2 & 99.9 & 0.835 & 1.720 & 0.1 & 0.0 & 20.0 & 0.0 & 5.1\tabularnewline
 & HM{[}$\mathcal{H},\mathcal{H}${]} & N/A & 48.9 & 61.2 & 30.3 & 56.5 & 0.0 & 100.0 & 0.4 & 99.9 & 0.830 & 1.752 & 0.2 & 0.0 & 19.5 & 0.0 & 5.0\tabularnewline
\end{tabular}}

\caption{Full data table for desth matrix.}
\label{tab:srcdest}
\end{table*}


Complete experimental results for ``Tab.~2: Combinations of Source \&
Destination Hardness. \ldots'' can be found in \cref{tab:srcdest} of this
supplementary material.

~\newline
\noindent\textbf{IV. Additional Notes on the Experimental Results}

\noindent\ul{(1) Slight ERS decrease with a larger $\eta$}

In some cases, \eg, HM[$\mathcal{S},g_\mathsf{LGA}$]\&ICS on the CARS dataset
reaches a slightly lower ERS with $\eta=32$ compared to that with $\eta=8$.
%
We speculate this is because adversarial training suffers from overfitting~\cite{bagoftricks,overfitting}
on adversarial examples, as mentioned in Sec.~2.

\subsection{Potential Future Work}

\noindent\textbf{I. Faster Adversarial Training with HM}

As discussed in Sec.~3.1, one potential adversarial training acceleration
method is to incorporate Free Adversarial Training (FAT)~\cite{freeat}
(originally for classification) into our DML adversarial training with HM.
%
Besides, directly incorporating FAT into the min-max adversarial training of
DML will easily result in model collapse as well, because the FAT algorithm can
be interpreted as to maintain a universal (agnostic to sample) perturbation
that can maximize the loss.
%
Thus, non-trivial modifications are still required to incorporate FAT FAT into
adversarial training with HM. This is left for future work.

~\newline\textbf{II. Better Choice for Destination Hardness}

The proposed LGA function incorporates our empirical observation that
``adversarial triplets should remain Semihard'' based on the results in Tab.~2.
%
However, a better choice for $H_\mathsf{D}$ may exist between ``Semihard'' and
``Softhard'' that can achieve better overall performance.
%
In the manuscript, we only use the existing sampling methods and simple
pseudo-hardness functions in order to avoid distraction from our focus.

~\newline\textbf{III. Other Loss Functions}

Adversarial trainig with DML loss functions other than triplet loss is
insufficiently explored.
%
New metric learning loss functions oriented for adversarial training are also
left for future study.
%
Besides, model collapse is an inevitable problem for adversarial training
with triplet loss, and it is unknown whether other loss functions could
mitigate this issue.

~\newline\textbf{IV. DML \& Classification}

It is unknown whether DML defenses will improve the robustness in the
classification task.

% overfitting is not observed... the performance simply increases slowly.
